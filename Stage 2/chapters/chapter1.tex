\chapter{Introduction}
In the rapid development of the information era, many daily events are achieved by a various of electronic devices such as computer or phones.
Those electronic devices highly rely on integrated circuits(ICs) to perform specific events. For example, bank transaction can be done by different devices,
the process contain personal data, and including usage of sensitive information. Therefore, information security like authentication, protecting confidential data
has become important in nowadays society. In order to increase security's robustness, a range of ways has been proposed. One conventional way to is by storing secret key in non-volatile memory to encrypt sensitive data with it,
and use asymmetric cryptography to authenticate the device [3]. However, the implementation process of cryptography is expensive, especially on resource-constraint device, and the device is still vulnerable to invasion attack. Ideally, devices should be able to handle challenging problems corresponding to energy consumption, 
computational power and the ability to fight against cyber attack. \par

PUF(physical unclonable function) has the ability to deal with these challenges. It does not store secret in non-volatile memory, instead, the volatile secret is derived from devices' physical characteristics [3] .
This is based on the inevitable random variation in ICs manufacturing process, which leads to the fact that no two IC have exact same physical characteristic. For example, each ICs has unique delay sequences in the transistors and wires.
With this property, PUF does not require lots of computational power and is cost-effective because no need to implement cryptographic operation, which works particular well on
resources-constraint devices such as RFID. Also, the attacker needs to perform attack when the device is on, which significantly increase the difficulty. As for invasion attack, 
the attacker needs to have the exact information of its unique physical characteristics to successfully derive secret. Overall, PUF provides another interesting way for reinforce security.

\section{Aims and Objectives}
The objectives split into two parts for this project. In the first stage, propose a novel machine learning to modeling different PUF(physical unclonable function)
behavior, so predict the response from a given PUF when given challenge bits. For example, considered the simplest PUF which is arbiter PUF, its operation to create a response is to
input a challenge bits(binary), and two signals will go thorough the multiplexers in the PUF structure depend on the value of it. Consequently response a binary bit
that will indicate which signal is faster. Therefore, the machine learning for modeling will be related to ETA(estimated time arrival) problem since the structure of PUF is similar to a road network. 
For instance, traveling through each multiplexer is similar to traveling through each road segment, and both of them have delay to affect the time of arrival. Overall the first stage is to design a machine learning
consider these concepts. In the second stage, design a reconfigurability framework to fight against such modeling. In detail, evaluate the machine learning by insert noise in PUF or experiment on OPUF(one-time-PUF) which
contain reconfiguration process that can alleviate modeling attack.

\section{Overview of the Report}
The remaining of the paper will organize as follow. Chapter 2 provide literature survey of the concept of PUF, including PUF's properties, detailed circuit structure and operational process, 
exist modeling attack with experiment results and application in life. Relative machine learning idea will be discuss as well. Chapter 3 describe the aim and objectives for the project
, gives in-depth analyzes how the project will be evaluated, the tests and experiments that support this. Chapter 4 demonstrate the current progress on the project. Chapter 5 provide brief
summary on the main achievements with a well organized future plan for the project.
