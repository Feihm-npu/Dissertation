\chapter{Introduction}

\subsection{Conventional protection of devices}
In the rapid development of the information era, many daily events are achieved by a variety of electronic devices such as computers or phones.
Those electronic devices highly rely on integrated circuits(ICs) to perform specific events. For example, bank transactions can be done by different devices,
the process contains personal data, and includes the usage of sensitive information. Therefore, information security like authentication, protecting confidential data
has become important in nowadays society. In order to increase security's robustness, a range of ways has been proposed. One conventional way is by storing secret keys in non-volatile memory to encrypt sensitive data with it
and using asymmetric cryptography to authenticate the device \cite{Reference3}. However, the implementation process of cryptography is expensive, especially on the resource-constraint device, and the device is still vulnerable to invasion attacks. Ideally, devices should be able to handle challenging problems corresponding to energy consumption, 
computational power and the ability to fight against cyber attacks. \par

\subsection{Advantage of PUF compare to conventional protection on devices}
PUF(physical unclonable function) has the ability to deal with the challenges mentioned at section 1.0.1. It does not store secrets in non-volatile memory, instead, the volatile secret is derived from devices' physical characteristics \cite{Reference3}.
This is based on the inevitable random variation in the ICs manufacturing process, which leads to the fact that no two ICs have exact same physical characteristics \cite{Reference3}. For example, each ICs has unique delay sequences in the transistors and wires.
With this property, PUF does not require lots of computational power and is cost-effective because no need to implement the cryptographic operation, which works particularly well on
resources-constraint devices such as RFID. Also, the attacker needs to perform an attack when the device is on, which significantly increase the difficulty. As for an invasion attack, 
the attacker needs to have the exact information of its unique physical characteristics to successfully derive secrets \cite{Reference3}. Overall, PUF provides another interesting way of reinforcing security.

\section{Aims and Objectives of the project}
The objectives are split into two parts for this project. In the first stage, propose a novel machine learning to model different PUF(physical unclonable function)
behaviour, so predict the response from a given PUF when given challenge bits. For example, considered the simplest PUF which is arbiter PUF, its operation to create a response is to
input a challenge bits(binary), and two signals will go through the multiplexers in the PUF structure depending on the value of it. Consequently response a binary bit
that will indicate which signal is faster. Therefore, the machine learning for modeling will be related to the ETA(estimated time arrival) problem since the structure of PUF is similar to a road network. 
For instance, travelling through each multiplexer is similar to travelling through each road segment, and both of them have delays to affect the time of arrival. Overall the first stage is to design a modeling attack
considering these concepts. In the second stage, design a reconfigurability framework to fight against such modeling. In detail, evaluate the modeling attack by inserting noise in PUF or experiment on OPUF(one-time-PUF) which
contain a reconfiguration process that can alleviate modeling attack.

\section{Overview of the Report}
The remaining of the paper will organize as follow. Chapter 2 provide a literature survey of the concept of PUF, including PUF's properties, detailed circuit structure and authentication process, 
existing modeling attack with experiment results, and reconfigurability framework. Chapter 3 describe the aim and objectives for the project, gives in-depth analyzes of how the project will be evaluated, 
the tests and experiments that support this. Chapter 4 demonstrate the current progress along with an explanation of the project. Chapter 5 provide brief summary of the main achievements with a well-organized future plan for the project.